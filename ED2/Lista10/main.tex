\documentclass{article}

\usepackage[utf8]{inputenc}
\usepackage[bottom=2cm, top=2cm, left=2cm, right=2cm]{geometry}
\usepackage{amsmath}
\usepackage{tikz}
\usetikzlibrary{positioning}

\title{Lista 10 - DFS}
\author{Vinícius Couto Tasso}
\date{}

\begin{document}

\maketitle
         
\begin{enumerate}

\item Considere a execução da busca em profundidade (DFS) começando pelo vértice A. Assuma que as listas de adjacências estão em ordem lexográfica/alfabética, ou seja, ao explorar o vértice E, considere E-D antes de E-G ou E-H. Complete a lista de vértices da ordem de descoberta do DFS.

\begin{verbatim}
    A-B 
    B-C 
    C-E 
    E-D 
    D-G
    D-H
    H-G
    E-G
    E-H
    C-F
    F-E
    F-H
    F-I
    I-E
    I-H
    B-D
    B-E
    B-F
    A-D
    A-E
\end{verbatim}

\end{enumerate}

\end{document}
